\documentclass{irmaart}
% Time-stamp: <ha.sty 28 nov 2009 08h12m30>
\ProvidesPackage{ha}
[2009/11/08 ver0.8 \space Additional constructs for "Handbook of automata"]
%========== Caption =====================================
\usepackage[font=up,margin=10pt,labelfont=bf,labelsep=period]{caption}
%========== Equations =====================================
\numberwithin{equation}{section}
%========== Times =========================================
\usepackage{times}
%========== Index =========================================
\usepackage{makeidx}
\makeindex
%========== Statements =====================================
%--- a common counter for theorem, corollary, lemma, proposition,
% conjecture (all italicized) and remark (not italicized)
\theoremstyle{plain}
\newtheorem{theorem}{Theorem}[section]
\newtheorem{corollary}[theorem]{Corollary}
\newtheorem{lemma}[theorem]{Lemma}
\newtheorem{proposition}[theorem]{Proposition}
\newtheorem{conjecture}[theorem]{Conjecture}
\theoremstyle{definition}
\newtheorem{remark}[theorem]{Remark}
\newtheorem{observation}[theorem]{Observation}
%--- a new counter for definitions
\newtheorem{definition}{Definition}[section]
%--- a new counter for examples
\newtheorem{example}{Example}[section]
%========== Url =====================================
\usepackage{url}
\usepackage{xcolor}
%========== Vaucanson =====================================
%\usepackage{vaucanson-g}
%\newcommand\useVCPrefgastex{%
%%%%%%%%%%%%%%%%%%%%%%%%%%%%%%%
%%
%% Package `Vaucanson-G'  version 0.4
%%
%% This is file `VCPref-mystyle'.
%%
%%   The actual values are those that give figure a look like with 
%%   the GasTex package.
%%  
%%%%%%%%%%%%%%%%%%%%%%%%%%%%%%%
%%% Scales  --- 
%%%%%%%%%%%%%%%%%%%%%%%%%%%%%%%
\renewcommand{\LargeScale}{1.2}         %float : argument of a \psscalebox
\renewcommand{\MediumScale}{1}          %float
\renewcommand{\SmallScale}{.7}		%float
\renewcommand{\TinyScale}{0.5}		%float
%%%%%%%%%%%%%%%%%%%%%%%%%%%%%%%
%%% State parameters  --- Default settings
%%%%%%%%%%%%%%%%%%%%%%%%%%%%%%%
\setlength{\LargeStateDiameter}{1.2cm}		%length
\setlength{\MediumStateDiameter}{.8cm}		%length
\setlength{\SmallStateDiameter}{.6cm}		%length
\renewcommand{\StateDblDimen}{outer}
%%%%%%%%%%%%%%%%%%%%%%%%%%%%%%%
% State aspect
%%%%%%%%%%%%%%%%%%%%%%%%%%%%%%%%%
\SetStateLineWidth{.14mm}		%% length
\SetStateFillStatus{none}		%% aspect
\SetStateFillColor{black}		%% color
\SetStateLabelScale{1}                  %% float
\FixStateLineDouble{2}{5}               %% Double style: 
%%%%%%%%%%%%%%
% Edge aspect
%%%%%%%%%%%%%%
\SetEdgeLineWidth{.14mm}			%% length
\SetEdgeLabelColor{black}		%% color
\SetEdgeLabelScale{1}		        %% float
\FixEdgeLineDouble{1.5}{2}		%% float : 
%%% arrows
\SetEdgeArrowWidth{1.03mm}			%width of the edge arrow
\SetEdgeArrowLengthCoef{1.37}                   %float : 
\setlength{\EdgeDblArrowWidth}{1.3mm}		%width : 
\renewcommand{\EdgeDblArrowLengthCoef}{1.09}	% 
\SetEdgeArrowInsetCoef{0}		%float : coef*\EdgeArrowSizeDim
%%%%%%%%%%%%%%%%%%%%%%%%%%%%%%%
% Arc geometry
%%%%%%%%%%%%%%%%%%%%%%%%%%%%%%%
\SetArcAngle{17}			%% int (degree)
\SetLArcAngle{30}			%% int (degree)
\SetArcCurvature{0.7}			%% float
\SetArcOffset{1pt}			%% length
%%%%%%%%%%%%%%%%%%%%%%%%%%%%%%%
% Loop geometry
%%%%%%%%%%%%%%%%%%%%%%%%%%%%%%%
\renewcommand{\LoopOnLargeState}{5.5} 		%float
\renewcommand{\LoopOnMediumState}{7}		%float : curvature
\renewcommand{\LoopOnSmallState}{9} 		%float
\renewcommand{\LoopOnVariableState}{4.5} 		%float
%%%%%%%%%%%%%%%%%%%%%%%%%%%%%%%
%%% Edge labels positionning
%%%%%%%%%%%%%%%%%%%%%%%%%%%%%%%
\renewcommand{\EdgeLabelPosit}{.5}   %per cent 
\renewcommand{\EdgeLabelRevPosit}{.5}
\renewcommand{\ArcLabelPosit}{.5}
\renewcommand{\ArcLabelRevPosit}{.5}
\renewcommand{\LArcLabelPosit}{.5}
\renewcommand{\LArcLabelRevPosit}{.5}
\renewcommand{\LoopLabelPosit}{.5}
\renewcommand{\LoopLabelRevPosit}{.5}
\renewcommand{\CLoopLabelPosit}{.5}
\renewcommand{\CLoopLabelRevPosit}{.5}
%%%%%%%%%%%%%%%%%%%%%%%%%%%%%%%
%%% Initial states parameters
%%%%%%%%%%%%%%%%%%%%%%%%%%%%%%%
\renewcommand{\ArrowOnMediumState}{1}		%float 
\renewcommand{\ArrowOnSmallState}{1} 		%float
\renewcommand{\ArrowOnLargeState}{1}		%float
%%%%%%%%%%%%%%%%%%%%%%%%%%%%%%%
}

%   \useVCPrefgastex
%========== Gastex =====================================
%\usepackage{gastex}
%========== Ulem =====================================
\usepackage[normalem]{ulem}

%\Red{\sout{Barred text}}
%\Blue{\xout{Hatched text}}
%\Red{\uwave{Wavelet underlined text}}

\newcommand{\Red}[1]{\textcolor{red}{#1}}
\newcommand{\Blue}[1]{\textcolor{blue}{#1}}

%========== Numbers ================================
\newcommand{\N}{\mathbb{N}}
\newcommand{\Z}{\mathbb{Z}}
\newcommand{\Q}{\mathbb{Q}}
\newcommand{\R}{\mathbb{R}}
\newcommand{\C}{\mathbb{C}}

%========== New Table of contents ===================
\makeatletter
\renewcommand\l@section[2]{%
  \ifnum \c@tocdepth >\z@
    \addpenalty\@secpenalty
    \addvspace{0em \@plus\p@}%modif
    \setlength\@tempdima{1.5em}%
    \begingroup
      \parindent \z@ \rightskip \@pnumwidth
      \parfillskip -\@pnumwidth
%      \leavevmode \bfseries
      \advance\leftskip\@tempdima
      \hskip -\leftskip
      #1\nobreak\hfil \nobreak\hb@xt@\@pnumwidth{\hss #2}\par
    \endgroup
  \fi}
\newcommand\localtableofcontents{%
    \section*{\contentsname}%
    \@starttoc{toc}%
    %\vfill
    }
%---------- modification of the maketitle of irmaart.cls ------
\renewcommand\@maketitle{%
  \newpage
%  \null
%  \vskip 2em%
  \begin{center}%
    {\Large\bfseries\boldmath \@title \par} 
    \vskip 24\p@%
        {\itshape\large
      \lineskip .5em%
      \begin{tabular}[t]{c}%
        \large\@author
      \end{tabular}\par}%
     \vskip9pt
     {\small%\itshape  <------- here itshape removed
     \begin{tabular}[t]{c}
     \@address
     \end{tabular}\par}
     \end{center}%
  \par
  \vskip 60\p@}
%========== Keywords and subject classification =============================
\renewenvironment{classification}{\vspace{.5cm}\noindent\small 2010
  Mathematics Subject Classification:}{\vskip 6\p@}
%========== Lists ==================================
\def\@listI{\leftmargin\leftmargini
            \parsep 0\p@ \@plus1\p@ \@minus\p@
            \topsep 2\p@ \@plus1\p@ \@minus\p@
            %\setlength{\partopsep}{0pt}%
            \itemsep 0\p@}
\let\@listi\@listI
\@listi
\def\@listii {\leftmargin\leftmarginii
              \labelwidth\leftmarginii
              \advance\labelwidth-\labelsep
              \topsep    0\p@ \@plus\p@ \@minus\p@}
\def\@listiii{\leftmargin\leftmarginiii
              \labelwidth\leftmarginiii
              \advance\labelwidth-\labelsep
              \topsep    0\p@ \@plus\p@\@minus\p@
              \parsep    \z@
              \partopsep \p@ \@plus\z@ \@minus\p@}
\makeatother
%%%%%%%%%%%%%%%%%%%%%%%%%%%%%%%%%%%%%%%%%%%%%%%%%%%%%%%%%%%
%                                                         %
%   Trois environnements pour enumerer des conditions :   %
%   conditions : (1), (2), (3), ...                       %
%   conditionsabc : (a), (b), (c), ...                    %
%   conditionsiii : (i), (ii), (iii), ...                 %
%                                                         %
%%%%%%%%%%%%%%%%%%%%%%%%%%%%%%%%%%%%%%%%%%%%%%%%%%%%%%%%%%%

\renewcommand{\labelenumi}{(\theenumi)}

%%%%%%%%%%%%%%%%%%%%%%%%%%%%%%%%%%%%%%%
\makeatletter
\newenvironment{conditions}
{%
  \begin{list}{\rm (\theenumi)}%
  {\noindent%
    \usecounter{enumi}%
    \topsep 2\p@ \@plus1\p@ \@minus\p@%\setlength{\topsep}{2pt}%
    %\setlength{\partopsep}{0pt}%
    \itemsep0\p@%\setlength{\itemsep}{2pt}%
    \parsep 0\p@ \@plus1\p@ \@minus\p@%\setlength{\parsep}{0pt}%
    \setlength{\leftmargin}{2.5em}%
    \setlength{\labelwidth}{1.5em}%
    \setlength{\labelsep}{0.5em}%
    \setlength{\listparindent}{0pt}%
    \setlength{\itemindent}{0pt}%
  }%
}%
{\end{list}}%

%%%%%%%%%%%%%%%%%%%%%%%%%%%%%%%%%%%%%%%%%%
\newenvironment{conditionsabc}
{%
  \begin{list}{\rm (\alph{enumi})}%
  {\noindent%
    \usecounter{enumi}%
    \topsep 2\p@ \@plus1\p@ \@minus\p@%\setlength{\topsep}{2pt}%
    %\setlength{\partopsep}{0pt}%
    \itemsep0\p@%\setlength{\itemsep}{2pt}%
    \parsep 0\p@ \@plus1\p@ \@minus\p@%\setlength{\parsep}{0pt}%
    \setlength{\leftmargin}{2.5em}%
    \setlength{\labelwidth}{1.5em}%
    \setlength{\labelsep}{0.5em}%
    \setlength{\listparindent}{0pt}%
    \setlength{\itemindent}{0pt}%
  }%
}%
{\end{list}}%

% %%%%%%%%%%%%%%%%%%%%%%%%%%%%%%%%%%%%%%%%%%
\newenvironment{conditionsiii}
{%
  \begin{list}{\rm (\roman{enumi})}%
  {\noindent%
    \usecounter{enumi}%
    \topsep 2\p@ \@plus1\p@ \@minus\p@%\setlength{\topsep}{2pt}%
    %\setlength{\partopsep}{0pt}%
    \itemsep0\p@%\setlength{\itemsep}{2pt}%
    \parsep 0\p@ \@plus1\p@ \@minus\p@%\setlength{\parsep}{0pt}%
    \setlength{\leftmargin}{2.5em}%
    \setlength{\labelwidth}{1.5em}%
    \setlength{\labelsep}{0.5em}%
    \setlength{\listparindent}{0pt}%
    \setlength{\itemindent}{0pt}%
 }%
}%
{\end{list}}%

 \makeatother

\let\le\leqslant
\let\ge\geqslant
\let\leq\leqslant
\let\geq\geqslant


%========== gastex =====================================
\usepackage{gastex}
%========== vaucanson =====================================
\usepackage{vaucanson-g}
\ChgStateLineWidth{0.5}
\ChgEdgeLineWidth{0.5}

\newcommand{\SqrState}[3][]%
 {\StateStyle %
  \psset{framearc=0}%
  \settowidth{\VariableStateWidth}{\scalebox{\StateLabelSca}{\scalebox{\StateLabelScale}{$#1$}}}%
  \addtolength{\VariableStateWidth}{\ExtraSpace}
  \ifthenelse{\lengthtest{\VariableStateWidth < \VariableStateIntDiam}}%
        {\setlength{\VariableStateWidth}{\VariableStateIntDiam}}{}%
  \setlength{\VariableStateITPos}{\ArrowOnStateCoef\StateDiam}%
  \addtolength{\VariableStateITPos}{0.5\VariableStateWidth}%
  \addtolength{\VariableStateITPos}{-0.5\StateDiam}%
  \rput#2{\pnode(\VariableStateITPos,0){#3e}%
          \pnode(-\VariableStateITPos,0){#3w}%
          \pnode(0,\ArrowOnStateCoef\StateDiam){#3n}%
          \pnode(0,-\ArrowOnStateCoef\StateDiam){#3s}}%
  \rput#2{\rnode{#3}{\psframebox{\protect\rule[-.5\VariableStateIntDiam]{0pt}{\VariableStateIntDiam}\protect\rule{\VariableStateWidth}{0pt}}}}
  \rput#2{\VaucStateRBLabel{#1}}%
}%
%========== Shows labels refs linenumbers ==============
\usepackage[displaymath]{lineno}
\usepackage[notcite]{showkeys}
     \linenumbers
%========== Hyperref at the end ========================
\usepackage[hypertex,hyperindex,pagebackref,final]{hyperref}
%========== Time calculation ==============================
\usepackage{calc}
\newcounter{hours}\newcounter{minutes}
\newcommand\printtime{\setcounter{hours}{\time/60}%
  \setcounter{minutes}{\time-\value{hours}*60}%
  \thehours\,h\,\theminutes}
\newcommand\dateandtime{\today\quad\printtime}
%========== Algorithms ====================================
\usepackage[boxed]{algorithm}    % pour les floats algorithmes
\usepackage[noend]{algorithmic}  % pour les algorithmes
  \newlength\commentspace
  \setlength{\commentspace}{3cm}
  \newcommand\algcomment[2]{\makebox[0pt][l]{\hspace{-#1em}%
    \hspace{\commentspace}$\triangleright$ #2}}
\newcommand{\algorithmicfunc}[1]{\textsc{#1}}
\newcommand{\FUNC}[1]{\item[\algorithmicfunc{#1}]}
%========== PSTricks =====================================
\usepackage{pst-all}
 \newpsobject{showgrid}{psgrid}{%
  subgriddiv=1,griddots=10,gridlabels=6pt}
%========== Proofs =====================================
%------------- end of proof in equation -------
\newcommand{\eqed}{\tag*{\qedsymbol}}
%========== Math notations ===================================
\newcommand{\Card}{\operatorname{Card}}
\newcommand\A{\mathcal{A}}
\newcommand\B{\mathcal{B}}
\newcommand\Tau{T}
\newcommand{\cF}{\mathcal{F}}
\newcommand{\cM}{\mathcal{M}}
\newcommand{\cP}{\mathcal{P}}
\newcommand{\cQ}{\mathcal{Q}}
\newcommand{\cS}{\mathcal{S}}
\newcommand{\cT}{\mathcal{T}}
\newcommand{\cU}{\mathcal{U}}
\newcommand{\cW}{\mathcal{W}}
\newcommand\e{\varepsilon}

%========== Shortcuts =============================
\newcommand{\sa}{synchronizing au\-tom\-a\-ta}
\newcommand{\san}{synchronizing au\-tom\-a\-ton}
\newcommand{\sw}{reset word}
\newcommand{\sws}{reset words}

%============Additional statements ==================
\theoremstyle{plain}
\newtheorem{question}[theorem]{Question}
%========== Hyphenation =============================

%========== For BiBTEX =============================
\def\Cerny{\v{C}ern\'y}

%====================================================
\begin{document}
%========== Headers =============================
\markboth{J.~Kari, M.~Volkov}{\v{C}ern\'{y}'s conjecture and the road coloring problem}

%=====================================================
\title{\v{C}ern\'{y}'s conjecture and the road coloring problem}
\author{Jarkko Kari$^1$,
  Mikhail Volkov$^2$}
\address{$^1$Department of Mathematics\\
FI-20014 University of Turku\\
Turku, Finland\\[2mm]
$^2$Department of Mathematics and Mechanics\\
620083 Ural State University\\
Ekaterinburg, Russia\\
email:\,\url{jkari@utu.fi,Mikhail.Volkov@usu.ru}\\[4mm]
\upshape{\dateandtime}}

%=====================================================
\maketitle\label{chapterKV}
%=====================================================

\begin{classification}
  68Q45 68R10
\end{classification}

\begin{keywords}
  Finite automata, Synchronizing automata, Reset words, \v{C}ern\'{y}'s conjecture,
  Road Coloring Problem
\end{keywords}



%==================================================
%Abstract
%\input{0-Abstract}
%Table of contents
\localtableofcontents

%=====================================================
\section{Synchronizing automata, their origins and importance}
%=====================================================
A complete deterministic finite automaton (DFA) $\mathcal{A}=(Q,A)$ (here and
below $Q$ stands for the state set and $A$ for the input alphabet) is called
\emph{synchronizing}\index{automaton!synchronizing} if there exists a word
$w\in A^*$ whose action resets $\mathcal{A}$, that is, $w$ leaves the automaton
in one particular state no matter at which state in $Q$ it is applied: $q\cdot
w=q'\cdot w$ for all $q,q'\in Q$. Any word $w$ with this property is said to be
a \emph{reset}\index{reset word} word for the automaton.

\begin{figure}[ht]
\unitlength=.75mm
\begin{center}
\begin{picture}(50,40)(-15,-10)
\gasset{Nh=6,Nw=6,Nmr=3,loopdiam=6} \node(A)(0,20){0}
\node(B)(20,20){1} \node(C)(20,0){2} \node(D)(0,0){3}
\drawedge(A,B){$a,b$} \drawedge(B,C){$b$} \drawedge(C,D){$b$}
\drawedge(D,A){$b$} \drawloop[loopangle=45](B){$a$}
\drawloop[loopangle=-45](C){$a$} \drawloop[loopangle=-135](D){$a$}
\end{picture}
\caption{The automaton $\mathcal{C}_4$}\label{KV:fig:C4}
\end{center}
\end{figure}
Figure~\ref{KV:fig:C4} shows a \san\ with 4~states denoted by $\mathcal{C}_4$.
The reader can easily verify that the word $ab^3ab^3a$ resets the automaton
leaving it in the state 1. With somewhat more effort one can also check that
$ab^3ab^3a$ is the shortest reset word for $\mathcal{C}_4$. The example in
Figure~\ref{KV:fig:C4} is due to \v{C}ern\'{y}, a Slovak computer scientist, in
whose pioneering paper~\cite{Cerny:1964} the notion of a \san\ explicitly
appeared for the first time. (\v{C}ern\'{y} called such automata
\emph{directable}.  The word \emph{synchronizing} in this context was probably
introduced by Hennie~\cite{Hennie:1964}.) \marginpar{\textbf{Needs
double-checking!!}} Implicitly, however, this concept has been around since the
earliest days of automata theory. The very first \san\ that we were able to
trace back in the literature appeared in Ashby's classic
book~\cite[pp.\,60--61]{Ashby:1956}, see \cite[Section~1]{Volkov:2008} for a
discussion.

In~\cite{Cerny:1964} the notion of a \san\ arose within the classic framework
of Moore's ``Gedanken-experiments''~\cite{Moore:1956}. For Moore and his
followers finite automata served as a mathematical model of devices working in
discrete mode, such as computers or relay control systems. This leads to the
following natural problem: how can we restore control over such a device if we
do not know its current state but can observe outputs produced by the device
under various actions? Moore~\cite{Moore:1956} has shown that under certain
conditions one can uniquely determine the state at which the automaton arrives
after a suitable sequence of actions (called an \emph{experiment}). Moore's
experiments were adaptive, that is, each next action was selected on the basis
of the outputs caused by the previous actions. Ginsburg~\cite{Ginsburg:1958}
considered more restricted experiments that he called \emph{uniform}. A uniform
experiment\footnote{After \cite{Gill:1961}, the name \emph{homing sequence} has
become standard for the notion.} is just a fixed sequence of actions, that is,
a word over the input alphabet; thus, in Ginsburg's experiments outputs were
only used for calculating the resulting state at the end of an experiment. From
this, just one further step was needed to come to the setting in which outputs
were not used at all. It should be noted that this setting is by no means
artificial---there exist many practical situations when it is technically
impossible to observe output signals. (Think of a satellite which loops around
the Moon and cannot be controlled from the Earth while ``behind'' the Moon.)

The original ``Gedanken-experiments'' motivation for studying \sa\
is still of importance, and reset words are frequently applied in
model-based testing of reactive systems. See \cite{Cho&Jeong&Somenzi&Pixley:1993,
Boppana&Rajan&Takayama&Fujita:1999} as typical samples of technical
contributions to the area and \cite{Sandberg:2005} for a recent survey.

Another strong motivation comes from the coding theory. We refer
to \cite[Chapters~3 and~10]{Berstel&Perrin&Reutenauer:2009} for a
detailed account of profound connections between codes and
automata; here we restrict ourselves to a special (but still very
important) case of maximal prefix codes. Recall that a
\emph{prefix code}\index{prefix code} over a finite alphabet $A$
is a set $X$ of words in $A^*$ such that no word of $X$ is a
prefix of another word of $X$. A prefix code is
\emph{maximal}\index{prefix code!maximal} if it is not contained
in another prefix code over the same alphabet. A maximal prefix
code $X$ over $A$ is \emph{synchronized}\index{prefix
code!synchronized} if there is a word $x\in X^*$ such that for any
word $w\in A^*$, one has $wx\in X^*$. Such a word $x$ is called a
\emph{synchronizing word}\index{synchronizing word of a code} for
$X$. The advantage of synchronized codes is that they are able to
recover after a loss of synchronization between the decoder and
the coder caused by channel errors: in the case of such a loss, it
suffices to transmit a synchronizing word and the following
symbols will be decoded correctly. Moreover, since the probability
that a word $v\in A^*$ contains a fixed factor $x$ tends to 1 as
the length of $v$ increases, synchronized codes eventually
resynchronize by themselves, after sufficiently many symbols being
sent. (As shown in~\cite{Capocelli&Gargano&Vaccaro:1988}, the
latter property in fact characterizes synchronized codes.) The
following simple example illustrates these ideas: let $A=\{0,1\}$
and $X=\{000,0010,0011,010,0110,0111,10,110,111\}$. Then $X$ is a
maximal prefix code and one can easily check that each of the
words 010, 011110, 011111110, \dots\ is a synchronizing word for
$X$. For instance, if the code word 000 has been sent but, due to
a channel error, the word 100 has been received, the decoder
interprets 10 as a code word, and thus, loses synchronization.
However, with a high probability this synchronization loss only
propagates for a short while; in particular, the decoder
definitely resynchronizes as soon as it encounters one of the
segments 010, 011110, 011111110, \dots\ in the received stream of
symbols. A few samples of such streams are shown in
Figure~\ref{KV:fig:decoding} in which vertical lines show the
partition of each stream into code words and the boldfaced code
words indicate the position at which the decoder resynchronizes.
\begin{figure}[h]
\begin{center}
\begin{tabular}{ll}
Sent & $0\,0\,0\ \mid 0\,0\,1\,0\,\ \mid\mathbf{0\,1\,1\,1\mid\dots}$\\
\mathstrut Received & $1\,0\mid 0\,0\,0 \mid 1\,0 \mid\mathbf{0\,1\,1\,1\mid\dots}$\\
\hline
\mathstrut Sent & $0\,0\,0\mid 0\,1\,1\,1 \mid 1\,1\,0\mid 0\,0\,1\,1 \mid 0\,0\,0 \mid 1\,0 \mid\mathbf{1\,1\,0\mid \dots}$\\
\mathstrut Received & $1\,0\mid 0\,0\,1\,1 \mid 1\,1\,1 \mid 0\,0\,0\mid 1\,1\,0 \mid 0\,0\,1\,0 \mid\mathbf{1\,1\,0\mid \dots}$\\
\hline
\mathstrut Sent & $0\,0\,0\mid 0\,0\,0 \mid 1\,1\,1\mid\mathbf{1\,0\mid \dots}$\\
\mathstrut Received & $1\,0\mid 0\,0\,0 \mid 0\,1\,1\,1 \mid\mathbf{1\,0\mid \dots}$
\end{tabular}
\caption{Restoring synchronization}\label{KV:fig:decoding}
\end{center}
\end{figure}

If $X$ is a finite prefix code over an alphabet $A$, then its
decoding can be implemented by a deterministic automaton that is
defined as follows. Let $Q$ be the set of all proper prefixes of
the words in $X$ (including the empty word $\varepsilon$). For
$q\in Q$ and $a\in A$, define
\begin{displaymath}
q\cdot a =\begin{cases} qa & \text{if $qa$ is a proper prefix of a word of $X$}\,,\\
\varepsilon & \text{if $qa \in X$}\,.\end{cases}
\end{displaymath}
The resulting automaton $\mathcal{A}_X$ is complete whenever the code $X$
is maximal and it is easy to see that $\mathcal{A}_X$ is a \san\ if and only
if $X$ is a synchronized code. Moreover, a word $x$ is synchronizing for $X$
if and only if $x$ is a \sw\ for $\mathcal{A}_X$ and sends all states in $Q$
to the state $\varepsilon$. Figure~\ref{KV:fig:huffman} illustrates this construction
for the code $X=\{000,0010,0011,010,0110,0111,10,110,111\}$ considered above.
The solid/dashed lines correspond to (the action of) 0/1.
\begin{figure}[htbp]
\FixVCScale{0.4}
\VCDraw{%
\begin{VCPicture}{(0,-1)(16,9)}
\MediumState
\ChgEdgeArrowWidth{6.5pt}
\VCPut{(0,0)}{
 \RstStateFillColor
\SqrState[0010]{(2,0)}{14}%
\SqrState[0011]{(4,0)}{15}%
\SqrState[0110]{(6,0)}{16}%
\SqrState[0111]{(8,0)}{17}%
\StateVar[001]{(3,2)}{4}%
\SqrState[000]{(1,2)}{9}%
\StateVar[011]{(7,2)}{7}%
\SqrState[010]{(5,2)}{10}%
\SqrState[110]{(11,2)}{11}%
\SqrState[111]{(13,2)}{12}%
\State[00]{(2,4)}{3}%
\State[01]{(6,4)}{6}%
\State[11]{(12,4)}{8}%
\SqrState[10]{(10,4)}{13}%
\State[0]{(4,6)}{2}%
\State[1]{(11,6)}{5}%
\State[\varepsilon]{(8,8)}{1}%
\ChgEdgeLineWidth{2}
\ChgEdgeLineColor{black}
\ChgEdgeLineStyle{solid}
\EdgeL[.5]{1}{2}{}
\EdgeL[.5]{2}{3}{}
\EdgeL[.5]{5}{13}{}
\EdgeL[.5]{3}{9}{}
\EdgeL[.5]{6}{10}{}
\EdgeL[.5]{8}{11}{}
\EdgeL[.5]{4}{14}{}
\EdgeL[.5]{7}{16}{}
\ChgEdgeLineWidth{2}
\ChgEdgeLineColor{black}
\ChgEdgeLineStyle{dashed}
\EdgeL[.5]{1}{5}{}
\EdgeL[.5]{2}{6}{}
\EdgeL[.5]{8}{12}{}
\EdgeL[.5]{4}{15}{}
\EdgeL[.5]{7}{17}{}
\EdgeL[.5]{5}{8}{}
\EdgeL[.5]{3}{4}{}
\EdgeL[.5]{6}{7}{}
}
%%%%%%%%%%%%%%%%%%%%%%%%%%%%%%
\VCPut{(17,-2)}{
 \RstStateFillColor
\State[00]{(2,4)}{3}%
\State[01]{(6,4)}{6}%
\State[11]{(12,4)}{8}%
\StateVar[001]{(3,2)}{4}%
\StateVar[011]{(7,2)}{7}%
\State[0]{(4,6)}{2}%
\State[1]{(11,6)}{5}%
\State[\varepsilon]{(8,8)}{1}%
\ChgEdgeLineWidth{2}
\ChgEdgeLineColor{black}
\ChgEdgeLineStyle{solid}
\EdgeL[.5]{1}{2}{}
\EdgeL[.5]{2}{3}{}
\VCurveR[0.5]{angleA=90,angleB=2,ncurv=1.2}{5}{1}{}
\VArcR[.5]{arcangle=60,ncurv=0.7}{3}{1}{}
\ArcR[.5]{6}{1}{}
\ChgEdgeLineWidth{2}
\ChgEdgeLineColor{black}
\ChgEdgeLineStyle{dashed}
\EdgeL[.5]{1}{5}{}
\EdgeL[.5]{2}{6}{}
\ChgEdgeLineWidth{2}
\ChgEdgeLineColor{black}
\ChgEdgeLineStyle{solid}
\VCurveR[0.5]{angleA=140,angleB=140,ncurv=1.6}{4}{1}{}
\VCurveR[0.5]{angleA=20,angleB=50,ncurv=.9}{8}{1}{}
\VCurveR[0.5]{angleA=25,angleB=-74,ncurv=.6}{7}{1}{}
\ChgEdgeLineWidth{2}
\ChgEdgeLineColor{black}
\ChgEdgeLineStyle{dashed}
\EdgeL[.5]{3}{4}{}
\EdgeL[.5]{6}{7}{}
\EdgeL[.5]{5}{8}{}
\VCurveR[0.5]{angleA=170,angleB=120,ncurv=1.9}{4}{1}{}
\VCurveR[0.5]{angleA=10,angleB=76,ncurv=1.2}{8}{1}{}
\VCurveR[0.5]{angleA=15,angleB=-56,ncurv=.8}{7}{1}{}
}
\end{VCPicture}%
}
\caption{A synchronized code (on the left) and its automaton (on the right)}\label{KV:fig:huffman}
\end{figure}

Thus, \textbf{(to be continued and supplied by some historical
references).}

An additional source of problems related to \sa\ has come from
\emph{robotics} or, more precisely, from part handling problems
in industrial automation such as part feeding, fixturing, loading,
assembly and packing. Within this framework, the concept of a \san\
was again rediscovered in the mid-1980s by Natarajan \cite{Natarajan:1986,
Natarajan:1989} who showed how \sa\ can be used to design sensor-free
orienters for polygonal parts, see \cite[Section~1]{Volkov:2008} for
a transparent example illustrating Natarajan's approach in a nutshell.
Since the 1990s \sa\ usage in the area of robotic manipulation has grown
into a prolific research direction but it is fair to say that publications
in this area deal mostly with implementation technicalities. However,
amongst them there are papers of significant theoretical importance
such as \cite{Eppstein:1990,Goldberg:1993,Chen&Ierardi:1995}.

\marginpar{\textbf{To~be\\ checked\\ with\\ Jarrko!!}} Recently, it has been
realized that a notion that arose in studying of \emph{substitution systems} is
also closely related to \sa. A \emph{substitution}\index{substitution} on a
finite alphabet $X$ is a map $\sigma:X\to X^+$; the substitution is said to be
of \emph{constant length}\index{substitution!of finite length} if all words
$\sigma(x)$, $x\in X$, have the same length. One says that $\sigma$ satisfies
the \emph{coincidence condition}\index{coincidence condition} if there exist
positive integers $m$ and $k$ such that all words $\sigma^k(x)$ have the same
$m$-th letter. For an example, consider the substitution $\tau$ on
$X=\{0,1,2\}$ defined by $0\mapsto 11,\ 1\mapsto 12,\ 2\mapsto 20$. Calculating
the iterations of $\tau$ up to $\tau^4$ (see Figure~\ref{KV:fig:substituion}),
we observe that
\begin{figure}[h]
\begin{center}
$\begin{matrix}
0&\mapsto&11&\mapsto&1212&\mapsto&12201220&\mapsto&122020\mathbf{1}112202011\\
1&\mapsto&12&\mapsto&1220&\mapsto&12202011&\mapsto&122020\mathbf{1}120111212\\
2&\mapsto&20&\mapsto&2011&\mapsto&20111212&\mapsto&201112\mathbf{1}212201220
\end{matrix}$ \caption{A substitution satisfying the coincidence
condition} \label{KV:fig:substituion}
\end{center}
\end{figure}
$\tau$ satisfies the coincidence condition (with $k=4$, $m=7$).

The importance of the coincidence condition comes from the crucial fact
(established by Dekking~\cite{Dekking:1978}) that it is this condition that
completely characterizes the constant length substitutions which give rise to
dynamical systems measure-theoretically isomorphic to a translation on a
compact Abelian group, see \cite[Chapter~7]{PytheasFogg:2002} for a survey. For
us, however, the coincidence condition is primarily interesting as yet another
incarnation of synchronizability. Indeed, there is a straightforward bijection
between DFAs and constant length substitutions. Each DFA $\mathcal{A}=(Q,A)$
with $A=\{a_1,\dots,a_\ell\}$ defines a length $\ell$ substitution on $Q$ that
maps every $q\in Q$ to the word $(q\cdot a_1)\dots (q\cdot a_\ell)\in Q^+$.
(For instance, the automaton $\mathcal{C}_4$ in Figure~\ref{KV:fig:C4} induces
the substitution $0\mapsto 11,\ 1\mapsto 12,\ 2\mapsto 23,\ 3\mapsto 30$.)
Conversely, each substitution $\sigma:X\to X^+$ such that all words
$\sigma(x)$, $x\in X$, have the same length $\ell$ gives rise to a DFA for
which $X$ serves as the state set and which has $\ell$ input letters
$a_1,\dots,a_\ell$, say, acting on $X$ as follows: $x\cdot a_i$ is the symbol
in the $i$-th position of the word $\sigma(x)$. (For instance, the substitution
$\tau$ considered in the previous paragraph defines the automaton shown in
Figure~\ref{KV:fig:C3}.)
\begin{figure}[ht]
\unitlength=.75mm
\begin{center}
\begin{picture}(50,30)(-15,-10)
\gasset{Nh=6,Nw=6,Nmr=3,loopdiam=6} \node(A)(10,20){0} \node(C)(20,0){1}
\node(D)(0,0){2} \drawedge(A,C){$a_1,a_2$} \drawedge(C,D){$a_2$}
\drawedge(D,A){$a_2$} \drawloop[loopangle=-45](C){$a_1$}
\drawloop[loopangle=-135](D){$a_1$}
\end{picture}
\caption{The automaton induced by the substitution $0\mapsto 11,\ 1\mapsto 12,\
2\mapsto 20$}\label{KV:fig:C3}
\end{center}
\end{figure}
It is clear that under the described bijection substitutions satisfying the
coincidence condition correspond precisely to \sa, and moreover, given a
substitution, the number of iterations at which the coincidence first occurs is
equal to the minimum length of \sw\ for the corresponding automaton.

We mention in passing a purely algebraic framework\marginpar{\textbf{If space\\
permits!!}} within which \sa\ also appear in a natural way. One may treat DFAs
as unary algebras since each letter of the input alphabet defines a unary
operation on the state set. A \emph{term}\index{unary term} in the language of
such unary algebras is an expression $t$ of the form $x\cdot w$, where $x$ is a
variable and $w$ is a word over an alphabet $A$. An
\emph{identity}\index{identity of unary algebras} is a formal equality between
two terms. A DFA $\mathcal{A}=(Q,A)$ \emph{satisfies} an identity $t_1=t_2$,
where the words involved in the terms $t_1$ and $t_2$ are over $A$, if $t_1$
and $t_2$ take the same value under each interpretation of their variables in
the set $Q$. Identities of unary algebras can be of the from either $x\cdot
u=x\cdot v$ (\emph{homotypical} identities\index{identity of unary
algebras!homotypical}) or $x\cdot u=y\cdot v$ with $x\ne y$
(\emph{heterotypical} identities\index{identity of unary
algebras!heterotypical}). It is easy to realize that a DFA is synchronizing if
and only if it satisfies a heterotypical identity, and thus, studying \sa\ may
be considered as a part of the equational logic of unary algebras. In
particular, \sa\ over a fixed alphabet form a \emph{pseudovariety} of unary
algebras. See \cite{Bogdanovic&Imreh&Ciric&Petkovic:1999} for a survey of
numerous publications in this direction; it is fair to say, however, that so
far this algebraic approach has not proved to be really useful for
understanding the combinatorial nature of \sa.


\section{Algorithmic and complexity issues}

It should be clear that not every DFA is synchronizing. Therefore,
the very first question that we should address is the following
one: \emph{given an automaton $\mathcal{A}$, how to determine
whether or not $\mathcal{A}$ is synchronizing?}

\begin{figure}[htb]
\begin{center}
\unitlength=.75mm
\begin{picture}(85,80)(-10,25)
\gasset{Nh=6,Nw=6,Nmr=3,loopdiam=6} \node(A)(-20,57){0}
\node(B)(0,57){1} \node(C)(0,38){2} \node(D)(-20,38){3}
\drawedge(A,B){$a,b$} \drawedge(B,C){$b$} \drawedge(C,D){$b$}
\drawedge(D,A){$b$} \drawloop[loopangle=45](B){$a$}
\drawloop[loopangle=-45](C){$a$} \drawloop[loopangle=-135](D){$a$}
\node(AD)(20,38){03} \node(AB)(20,57){01} \node(BC)(40,57){12}
\node(CD)(40,38){23} \node(AC)(60,57){02} \node(BD)(80,57){13}
\drawloop[loopangle=0](CD){$a$} \drawloop[loopangle=0](BD){$a$}
\drawloop[loopangle=45](BC){$a$}
\drawedge[linewidth=.5,AHLength=2,ELside=r](CD,AD){$b$}
\drawedge[linewidth=.5,AHLength=2](AD,AB){$b$}
\drawedge(AB,BC){$b$}
\drawedge[linewidth=.5,AHLength=2](BC,CD){$b$}
\drawedge[linewidth=.5,AHLength=2,curvedepth=5](AB,B){$a$}
\gasset{Nadjust=w,Nadjustdist=1.5,Nh=6,Nmr=2}
\node(ABC)(40,76){012} \node(ABD)(70,76){013}
\node(BCD)(40,95){123} \node(ACD)(70,95){023}
\node(ABCD)(10,95){0123} \drawloop(ABCD){$b$} \drawloop(BCD){$a$}
\drawedge[linewidth=.5,AHLength=2](ABCD,BCD){$a$}
\drawedge[linewidth=.5,AHLength=2,ELside=r](BCD,ACD){$b$}
\drawedge[curvedepth=-5,ELside=r](ACD,BCD){$a$}
\drawedge[linewidth=.5,AHLength=2](ACD,ABD){$b$}
\drawedge(ABC,BCD){$b$}
\drawedge[linewidth=.5,AHLength=2,ELside=r](ABD,ABC){$b$}
\drawedge[linewidth=.5,AHLength=2,curvedepth=-5,ELside=r](ABC,BC){$a$}
\drawedge[curvedepth=5](ABD,BD){$a$}
\drawedge[curvedepth=5](AC,BC){$a$}
\drawedge[curvedepth=5](BD,AC){$b$} \drawedge(AC,BD){$b$}
\drawedge[curvedepth=-20,ELside=r](AD,BD){$a$}
\end{picture}
\caption{The power automaton
$\mathcal{P}(\mathcal{C}_4$)}\label{KV:fig:power automaton}
\end{center}
\end{figure}

This question is in fact quite easy, and the most straightforward solution to
it can be achieved via the classic subset construction by Rabin and
Scott~\cite{Rabin&Scott:1959}. Given a DFA $\mathcal{A}=(Q,A)$, we define its
\emph{subset automaton}\index{subset automaton} $\mathcal{P}(\mathcal{A})$ on
the set of the non-empty subsets of $Q$ by setting $P\cdot a=\{p\cdot a\mid
p\in P\}$ for each non-empty subset $P$ of $Q$ and each $a\in\A$. (Since we
start with a deterministic automaton, we do not need adding the empty set to
the state set of $\mathcal{P}(\mathcal{A})$.) Figure~\ref{KV:fig:power
automaton} presents the subset automaton for the DFA $\mathcal{C}_4$ shown in
Figure~\ref{KV:fig:C4}.

Now it is obvious that a word $w\in A^*$ is a reset word for the
DFA $\mathcal{A}$ if and only if $w$ labels a path in
$\mathcal{P}(\mathcal{A})$ starting at $Q$ and ending at a
singleton. (For instance, the bold path in
Figure~\ref{KV:fig:power automaton} represents the shortest reset
word $ab^3ab^3a$ of the automaton $\mathcal{C}_4$.) Thus, the
question of whether or not a given DFA $\mathcal{A}$ is
synchronizing reduces to the following reachability question in
the underlying digraph of the subset automaton
$\mathcal{P}(\mathcal{A})$: is there a path from $Q$ to a
singleton? The latter question can be easily answered by
breadth-first search,  see, e.g.,
\cite[Section~22.2]{Cormen&Leiserson&Rivest&Stein:2001}.

The described procedure is conceptually very simple but rather
inefficient because the power automaton $\mathcal{P}(\mathcal{A})$
is exponentially larger than $\mathcal{A}$. However, the following
criterion of synchronizability~\cite[Theorem~2]{Cerny:1964} gives
rise to a polynomial algorithm.
\begin{proposition}
\label{KV:prop:quadratic} A DFA $\mathcal{A}=(Q,A)$ is synchronizing if and
only if for every $q,q'\in Q$ there exists a word $w\in A^*$ such that $q\cdot
w=q'\cdot w$.
\end{proposition}

\begin{proof}
Of course, only sufficiency needs a proof. For this, take two states $q,q'\in
Q$ and consider a word $w_1$ such that $q\cdot w_1=q'\cdot w_1$. Then $|Q\cdot
w_1|<|Q|$. If $|Q\cdot w_1|=1$, then $w_1$ is a \sw\ and $\mathcal{A}$ is
synchronizing. If $|Q\cdot w_1|>1$, take two states $p,p'\in Q\cdot w_1$ and
consider a word $w_2$ such that $p\cdot w_2=q'\cdot w_2$. Then $|Q\cdot
w_1w_2|<|Q\cdot w_1|$. $|Q\cdot w_1w_2|=1$, then $w_1w_2$ is a \sw; otherwise
we repeat the process. Clearly, a \sw\ for $\mathcal{A}$ will be constructed in
at most $|Q|-1$ steps.
\end{proof}

One can treat Proposition~\ref{KV:prop:quadratic} as a reduction of the
synchronizability problem to a reachability problem in the subautomaton
$\mathcal{P}^{[2]}(\mathcal{A})$ of $\mathcal{P}(\mathcal{A})$ whose states are
\emph{couples} (2-element subsets) and singletons of $Q$. Since the
subautomaton has $\dfrac{|Q|(|Q|+1)}2$ states, breadth-first search solves this
problem in $O(|Q|^2\cdot|A|)$ time. This complexity bound assumes that no reset
word is explicitly calculated. If one requires that, whenever $\mathcal{A}$
turns out to be synchronizing, a reset word is produced, then the best of the
known algorithms (which is basically due to Eppstein
\cite[Theorem~6]{Eppstein:1990}, see also \cite[Theorem~1.15]{Sandberg:2005})
has an implementation that consumes $O(|Q|^3 + |Q|^2\cdot|A|)$ time and
$O(|Q|^2 + |Q|\cdot|A|)$ working space, not counting the space for the output
which is $O(|Q|^3)$.

For a \san, the subset automaton can be used to construct shortest
reset words which correspond to shortest paths from the whole
state set $Q$ to a singleton. Of course, this requires exponential
(of $|Q|$) time in the worst case. Nevertheless, there were
attempts to implement this approach, see, e.g.,
\cite{Rho&Somenzi&Pixley:1993,Trahtman:2006}. One may hope that,
as above, a suitable calculation in the ``polynomial''
subautomaton $\mathcal{P}^{[2]}(\mathcal{A})$ may yield a
polynomial algorithm. However, it is not the case, and moreover,
as we will see, it is very unlikely that any reasonable algorithm
may exist for finding shortest reset words in general \sa. In the
following discussion we assume the reader's acquaintance with some
basics of computational complexity (such as the definitions of the
complexity classes \textsf{NP} and \textsf{coNP}) that can be
found, e.g., in~\cite{Garey&Johnson:1979,Papadimitriou:1994}.

Consider the following decision
problem:\index{\textsc{Short-Reset-Word}}

\smallskip

\hangindent=\parindent \noindent\textsc{Short-Reset-Word:}
\emph{Given a \san\ $\mathcal{A}$ and a positive integer $\ell$,
is it true that $\mathcal{A}$ has a reset word of length $\ell$?}

\smallskip

Clearly, \textsc{Short-Reset-Word} belongs to the complexity class
\textsf{NP}: one can non-deterministically guess a word $w\in A^*$
of length $\ell$ and then check if $w$ is a reset word for
$\mathcal{A}$ in time $\ell|Q|$. Several
authors~\cite{Rystsov:1980,Eppstein:1990,Goralcik&Koubek:1995,Salomaa:2003,Samotij:2007}
have proved that \textsc{Short-Reset-Word} is \textsf{NP}-hard by
a polynomial reduction from \textsc{SAT} (the satisfiability
problem for a system of \emph{clauses}, that is, disjunctions of
literals). We reproduce here Eppstein's reduction
from~\cite{Eppstein:1990}.

Given an arbitrary instance $\psi$ of \textsc{SAT} with $n$
variables $x_1,\dots,x_n$ and $m$ clauses $c_1,\dots,c_m$, we
construct a DFA $\mathcal{A}(\psi)$ with 2 input letters $a$ and
$b$ as follows. The state set $Q$ of $\mathcal{A}(\psi)$ consists
of $(n+1)m$ states $q_{i,j}$, $1 \le i \le m$, $1 \le j \le n+1$,
and a special state $z$. The transitions are defined by
\begin{align*}
& q_{i,j}\cdot a =
\begin{cases}
    z \text{ if the literal $x_j$ occurs in $c_i$},\\
    q_{i,j+1} \text{ otherwise}
\end{cases} && \text{ for $1 \le i \le m$, $1 \le j \le n$;} \\
&q_{i,j}\cdot b =
\begin{cases}
    z \text{ if the literal $\neg x_j$ occurs in $c_i$},\\
    q_{i,j+1} \text{ otherwise}
\end{cases} && \text{ for $1 \le i \le m$, $1 \le j \le n$;} \\
&q_{i,n+1}\cdot a=q_{i,n+1}\cdot b =z && \text{ for $1\le i\le
m$;}\\  &z\cdot a=z\cdot b =z.&&
\end{align*}
Figure~\ref{KV:fig:A2_example} shows two automata of the form
$\mathcal{A}(\psi)$ build for the \textsc{SAT} instances
\begin{align*}
\psi_1&=\{x_1 \vee x_2 \vee x_3,\, \neg x_1 \vee x_2,\, \neg x_2
\vee x_3,\,\neg x_2 \vee \neg x_3\},\\
\psi_2&=\{x_1 \vee x_2,\,\neg x_1 \vee x_2,\, \neg x_2 \vee
x_3,\,\neg x_2 \vee \neg x_3\}.
\end{align*}
If at some state $q\in Q$ in Figure~\ref{KV:fig:A2_example} there
is no outgoing edge labelled $c\in\{a,b\}$, the edge
$q\stackrel{c}{\to}z$ is assumed (those edges are omitted to
improve readability). The two instances differ only in the first
clause: in $\psi_1$ it contains the literal $x_3$ while in
$\psi_2$ it does not. Correspondingly, the automata
$\mathcal{A}(\psi_1)$ and $\mathcal{A}(\psi_2)$ differ only by the
outgoing edge labelled $a$ at the state $q_{1,3}$: in
$\mathcal{A}(\psi_1)$ it leads to $z$ (and therefore, it is not
shown) while in $\mathcal{A}(\psi_2)$ it leads to the state
$q_{1,4}$ and is shown by the dashed line.

Observe that $\psi_1$ is satisfiable for the truth assignment
$x_1=x_2=0$, $x_3=1$ while $\psi_2$ is not satisfiable. It is not
hard to check that the word $bba$ resets $\mathcal{A}(\psi_1)$
while $\mathcal{A}(\psi_2)$ is reset by no word of length~3 but by
every word of length~4.

\begin{figure}[t]
\unitlength=.75mm
\begin{center}
\begin{picture}(120,85)(-100,-10)
\node(n478)(-50,0){$q_{1,2}$} \node(n479)(10,0){$q_{1,4}$}
\node(n480)(-80,0){$q_{1,1}$} \node(n481)(-20,0){$q_{1,3}$}
\node(n75)(-50,20){$q_{2,2}$} \node(n32)(-20,20){$q_{2,3}$}
\node(n41)(10,20){$q_{2,4}$} \node(n202)(-80,20){$q_{2,1}$}
\node(n42)(10,40){$q_{3,4}$} \node(n172)(-80,40){$q_{3,1}$}
\node(n14)(-50,40){$q_{3,2}$} \node(n472)(-20,40){$q_{3,3}$}
\node(n474)(-50,60){$q_{4,2}$} \node(n475)(10,60){$q_{4,4}$}
\node(n476)(-80,60){$q_{4,1}$} \node(n477)(-20,60){$q_{4,3}$}

\drawedge(n480,n478){$b$} \drawedge(n478,n481){$b$}
\drawedge[ELdist=1.1](n32,n41){$a,b$} \drawedge(n472,n42){$b$}
\drawedge[ELdist=1.1,ELside=r](n476,n474){$a,b$}
\drawedge[ELside=r](n474,n477){$a$}
\drawedge[ELside=r](n477,n475){$a$} \drawedge(n202,n75){$a$}
\drawedge(n75,n32){$b$} \drawedge[ELdist=1.1](n172,n14){$a,b$}
\drawedge(n14,n472){$a$}

\node[Nw=8.32,Nh=7.0,Nmr=0.0](n1310)(-65,70){$x_1$}
\node[Nw=8.32,Nh=7.0,Nmr=0.0](n1316)(-35,70){$x_2$}
\node[Nw=8.32,Nh=7.0,Nmr=0.0](n1318)(-5,70){$x_3$}
\node[Nw=8.32,Nh=7.0,Nmr=0.0](n1367)(-95,0){$c_1$}
\node[Nw=8.32,Nh=7.0,Nmr=0.0](n1368)(-95,20){$c_2$}
\node[Nw=8.32,Nh=7.0,Nmr=0.0](n1369)(-95,40){$c_3$}
\node[Nw=8.32,Nh=7.0,Nmr=0.0](n1370)(-95,60){$c_4$}

\node(n1646)(30,30){$z$}

\drawedge[dash={3.0 3.0}{0.0},curvedepth=6](n481,n479){$a$ in
$\mathcal{A}(\psi_2)$}
\drawedge[curvedepth=-6,ELside=r,ELdist=2.0](n481,n479){$b$}
\end{picture}
\end{center}
\caption{The automata $\mathcal{A}(\psi_1)$ and
$\mathcal{A}(\psi_2)$} \label{KV:fig:A2_example}
\end{figure}

In general, it is easy to see that $\mathcal{A}(\psi)$ is reset by
every word of length $n+1$ and is reset by a word of length $n$ if
and only if $\psi$ is satisfiable. Therefore assigning the
instance $(\mathcal{A}(\psi),n)$ of \textsc{Short-Reset-Word} to
an arbitrary $n$-variable instance $\psi$ of \textsc{SAT}, one
obtains a polynomial reduction of the latter problem to the
former. Since \textsc{SAT} is \textsf{NP}-complete and
\textsc{Short-Reset-Word} lies in \textsf{NP}, we obtain the
following.

\begin{proposition}
\label{KV:prop:complexity1} The problem \textsc{Short-Reset-Word} is
\textsf{NP}-complete.\qed
\end{proposition}

In fact, as observed by Samotij~\cite{Samotij:2007}, the above
construction yields slightly more\footnote{Actually, the reduction
in~\cite{Samotij:2007} is not correct but the result claimed can
be easily recovered as shown below.}. Consider the following
decision problem:\index{\textsc{Shortest-Reset-Word}}

\smallskip

\hangindent=\parindent \noindent \textsc{Shortest-Reset-Word:}
\emph{Given a \san\ $\mathcal{A}$ and a positive integer $\ell$,
is it true that the minimum length of a reset word for
$\mathcal{A}$ is equal to $\ell$?}

\smallskip

\noindent Assigning the instance $(\mathcal{A}(\psi),n+1)$ of
\textsc{Shortest-Reset-Word} to an arbitrary system $\psi$ of
clauses on $n$ variables, one sees that the answer to the instance
is ``Yes'' if and only if $\psi$ is not satisfiable. Thus, we have
a polynomial reduction from the negation of \textsc{SAT} to
\textsc{Shortest-Reset-Word} whence the latter problem is
\textsf{coNP}-hard. As a corollary, \textsc{Shortest-Reset-Word}
cannot belong to \textsf{NP} unless \textsf{NP}\,=\,\textsf{coNP}
which is commonly considered to be very unlikely. In other words,
even non-deterministic algorithms cannot decide the minimum length
of a reset word for a given \san\ in polynomial time.

The exact complexity of the problem \textsc{Shortest-Reset-Word}
has been recently determined by
Gawrychowski~\cite{Gawrychowski:2008} and, independently, by
Olschewski and Ummels~\cite{Olschewski&Ummels:2010}. It turns out
that the appropriate complexity class is \textsf{DP}
(\textsf{Difference Polynomial-Time}) introduced by Papadimitriou
and Yannakakis~\cite{Papadimitriou&Yannakakis:1984}; this class
consists of languages of the form $L_1\cap L_2$ where $L_1$ is a
language from \textsf{NP} and a $L_2$ is a language in
\textsf{coNP}. A ``standard'' \textsf{DP}-complete problem is
\textsc{SAT-UNSAT} whose instance is a pair of clause systems
$\psi,\chi$, say, and whose question is whether $\psi$ is
satisfiable and $\chi$ is unsatisfiable.

\begin{proposition}
\label{KV:prop:complexity2} The problem \textsc{Shortest-Reset-Word} is
\textsf{DP}-complete.\qed
\end{proposition}

Proposition~\ref{KV:prop:complexity2} follows from mutual
reductions between \textsc{Shortest-Reset-Word} and
\textsc{SAT-UNSAT} obtained
in~\cite{Gawrychowski:2008,Olschewski&Ummels:2010}.

The complexity class $\mathsf{P}^\mathsf{NP[log]}$ is defined as
the class of all problems that can be solved by a deterministic
polynomial-time Turing machine that has an access to an oracle for
an \textsf{NP}-complete problem, with the number of queries being
logarithmic in the size of the input. The class \textsf{DP} is
contained in $\mathsf{P}^\mathsf{NP[log]}$ (if fact, for every
problem in \textsf{DP} two oracle queries suffice) and the
inclusion is believed to be strict. Olschewski and
Ummels~\cite{Olschewski&Ummels:2010} have shown that the problem
of computing the minimum length of \sws\ (as opposed to deciding
whether it is equal to a given integer) is complete for the
functional analogue $\mathsf{FP}^\mathsf{NP[log]}$ of the class
$\mathsf{P}^\mathsf{NP[log]}$ (see \cite{Selman:1994} for a
discussion of functional complexity classes). Hence, this problem
appears to be even harder than deciding the minimum length of
\sws. Recently Berlinkov~\cite{Berlinkov:2010} has shown (assuming
\textsf{P}\,$\ne$\,\textsf{NP}) that no polynomial algorithm can
approximate the minimum length of \sws\ for a given \san\ within a
constant factor.

The problem of finding a \sw\ of minimum length (as opposed to
computing only the length without writing down the word itself)
may be even more difficult. From the quoted result
of~\cite{Olschewski&Ummels:2010} it follows that the problem is
$\mathsf{FP}^\mathsf{NP[log]}$-hard but its exact complexity is
not known yet.

We mention that Pixley, Jeong and Hachtel~\cite{Pixley&Jeong&Hachtel:1992}
suggested an heuristic polynomial algorithm for finding short \sws\ in \sa\
that was reported to perform rather satisfactory on a number of benchmarks
from~\cite{Yang:1991}; further polynomial algorithms yielding short (though not
necessarily shortest) \sws\ have been implemented by
Trahtman~\cite{Trahtman:2006} and Roman~\cite{Roman:2009,Roman:2009a}. Some
algorithms for finding \sws\ will be also discussed in the next section.

\section{Around the \v{C}ern\'{y} conjecture}

\paragraph*{The \v{C}ern\'{y} conjecture.} A very natural question to ask
is the following: \emph{given a positive integer $n$, how long can be \sws\ for
\sa\ with $n$ states?} \v{C}ern\'{y}~\cite{Cerny:1964} found a lower bound by
constructing, for each $n>1$, a \san\
$\mathcal{C}_n$\index{automaton!\v{C}ern\'{y}} with $n$ states and 2 input
letters whose shortest \sw\ has length $(n-1)^2$. We assume that the state set
of $\mathcal{C}_n$ is $Q=\{0,1,2,\dots,n-1\}$ and the input letters are $a$ and
$b$, subject to the following action on $Q$:
\begin{displaymath}
i\cdot a=\begin{cases}
i &\text{if } i>0,\\
1 &\text{if } i=0;
\end{cases}\quad
i\cdot b=i+1\!\!\pmod{n}.
\end{displaymath}
Our first example of \san\ (see Figure\,\ref{KV:fig:C4}) is, in
fact, $\mathcal{C}_4$. A generic automaton $\mathcal{C}_n$ is
shown in Figure\,\ref{KV:fig:cerny-n} on the left.

\begin{figure}[ht]
\begin{center}
\unitlength .45mm
\begin{picture}(72,76)(25,-86)
\gasset{Nw=16,Nh=16,Nmr=8,loopdiam=12} \node(n0)(36.0,-16.0){1}
\node(n1)(4.0,-40.0){$0$} \node(n2)(68.0,-40.0){2}
\node(n3)(16.0,-72.0){$n{-}1$} \node(n4)(56.0,-72.0){3}
\drawedge[ELdist=2.0](n1,n0){$a,b$}
\drawedge[ELdist=1.5](n2,n4){$b$}
\drawedge[ELdist=1.7](n0,n2){$b$}
\drawedge[ELdist=1.7](n3,n1){$b$}
\drawloop[ELdist=1.5,loopangle=30](n2){$a$}
\drawloop[ELdist=2.4,loopangle=-30](n4){$a$}
\drawloop[ELdist=1.5,loopangle=-90](n0){$a$}
\drawloop[ELdist=1.5,loopangle=210](n3){$a$} \put(31,-73){$\dots$}
\end{picture}
\begin{picture}(72,76)(-25,-86)
\gasset{Nw=16,Nh=16,Nmr=8} \node(n0)(36.0,-16.0){1}
\node(n1)(4.0,-40.0){$0$} \node(n2)(68.0,-40.0){2}
\node(n3)(16.0,-72.0){$n{-}1$} \node(n4)(56.0,-72.0){3}
\drawedge[ELdist=2.0](n1,n0){$b$}
\drawedge[ELdist=1.5](n2,n4){$b,c$}
\drawedge[ELdist=1.7](n0,n2){$b,c$}
\drawedge[ELdist=1.7](n3,n1){$b,c$}
\drawedge[ELdist=2.0](n1,n2){$c$} \put(31,-73){$\dots$}
\end{picture}
\end{center}
\caption{The DFA $\mathcal{C}_n$ and the DFA $\mathcal{W}_n$
induced by the actions of $b$ and $c=ab$}\label{KV:fig:cerny-n}
\end{figure}

The series $\mathcal{C}_n$ was rediscovered many times (see, e.g.,
\cite{Laemmel&Rudner:1969,Fischler&Tannenbaum:1970,Eppstein:1990,Frettloh&Sing2007}).
It is easy to see that the word $(ab^{n-1})^{n-2}a$ of length
$n(n-2)+1=(n-1)^2$ is a reset word for $\mathcal{C}_n$. There are several nice
proofs for \v{C}ern\'{y}'s result~\cite[Lemma~1]{Cerny:1964} that
$\mathcal{C}_n$ has no shorter \sws. Here we present a recent proof
from~\cite{Ananichev&Gusev&Volkov:2010}; it is based on a transparent idea and
reveals an interesting connection between \v{C}ern\'{y}'s automata
$\mathcal{C}_n$ and an extremal series of digraphs discovered in Wielandt's
classic paper~\cite{Wielandt:1950} (see Section~\ref{KV:sec:rcp}).

Let $w$ be a \sw\ of minimum length for $\mathcal{C}_n$. Since the
letter $b$ acts on $Q$ as a cyclic permutation, the word $w$
cannot end with $b$. (Otherwise removing the last letter gives a
shorter \sw.) Thus, we can write $w$ as $w = w'a$ for some
$w'\in\{a,b\}^*$ such that the image of $Q$ under the action of
$w'$ is precisely the set $\{0,1\}$.

Since the letter $a$ fixes each state in its image
$\{1,2,\dots,n-1\}$, every occurrence of $a$ in $w$ except the
last one is followed by an occurrence of $b$. (Otherwise $a^2$
occurs in $w$ as a factor and reducing this factor to just $a$
results in a shorter \sw.) Therefore, if we let $c=ab$, then the
word $w'$ can be rewritten into a word $v$ over the alphabet
$\{b,c\}$. The actions of $b$ and $c$ induce a new DFA on the
state set $Q$; we denote this induced DFA (shown in
Figure\,\ref{KV:fig:cerny-n} on the right) by $\mathcal{W}_n$.
Since $w'$ and $v$ act on $Q$ in the same way, the word $vc$ is a
\sw\ for $\mathcal{W}_n$ and brings the automaton to the state~2.

If $u\in\{b,c\}^*$, the word $uvc$ also is a \sw\ for
$\mathcal{W}_n$ and it also brings the automaton to~2. Hence, for
every $\ell\ge|vc|$, there is a path of length $\ell$ in
$\mathcal{W}_n$ from any given state $i$ to~2. In particular,
setting $i=2$, we conclude that for every $\ell\ge|w|$ there is a
cycle of length $\ell$ in $\mathcal{W}_n$. The underlying digraph
of $\mathcal{W}_n$ has simple cycles only of two lengths: $n$ and
$n-1$. Each cycle of $\mathcal{W}_n$ must consist of simple cycles
of these two lengths whence each number $\ell\ge|w|$ must be
expressible as a non-negative integer combination of $n$ and
$n-1$. Here we invoke the following well-known and elementary
result from arithmetics:

\begin{lemma}[{\mdseries\cite[Theorem 2.1.1]{RamirezAlfonsin:2005}}]
\label{KV:lemma:sylvester} If $k_1,k_2$ are relatively prime positive integers,
then $k_1k_2-k_1-k_2$ is the largest integer that is not expressible as a
non-negative integer combination of $k_1$ and $k_2$.\qed
\end{lemma}

Lemma~\ref{KV:lemma:sylvester} implies that
$|vc|>n(n-1)-n-(n-1)=n^2-3n+1$. Suppose that $|vc|=n^2-3n+2$. Then
there should be a path of this length from the state~1 to the
state~2. Every outgoing edge of~1 leads to~2, and thus, in the
path it must be followed by a cycle of length $n^2-3n+1$. No cycle
of such length may exist by Lemma~\ref{KV:lemma:sylvester}. Hence
$|vc|\ge n^2-3n+3$.

Since the action of $b$ on any set $S$ of states cannot change the
cardinality of $S$ and the action of $c$ can decrease the
cardinality by~1 at most, the word $vc$ must contain at least
$n-1$ occurrences of $c$. Hence the length of $v$ over $\{b,c\}$
is at least $n^2-3n+2$ and $v$ contain at least $n-2$ occurrences
of $c$. Since each occurrence of $c$ in $v$ corresponds to an
occurrence of the factor $ab$ in $w'$, we conclude that the length
of $w'$ over $\{a,b\}$ is at least $n^2-3n+2+n-2=n^2-2n$. Thus,
$|w|=|w'a|\ge n^2-2n+1=(n-1)^2$.

\smallskip

If we define the \emph{\v{C}ern\'{y} function}\index{\v{C}ern\'{y} function}
$\mathfrak{C}(n)$ as the maximum length of shortest reset words for \sa\ with
$n$ states, the above property of the series $\{\mathcal{C}_{n}\}$,
$n=2,3,\dotsc$, yields the inequality $\mathfrak{C}(n)\ge(n-1)^2$. The
\emph{\v{C}ern\'{y} conjecture}\index{\v{C}ern\'{y} conjecture} is the claim
that the equality $\mathfrak{C}(n)=(n-1)^2$ holds true.

In the literature, one often refers to \v{C}ern\'{y}'s paper \cite{Cerny:1964}
as the source of the \v{C}ern\'{y} conjecture. In fact, the conjecture was not
yet formulated in that paper. There \v{C}ern\'{y} only observed that
\begin{equation}
\label{first bound} (n-1)^2\le \mathfrak{C}(n)\le 2^n-n-1
\end{equation}
and concluded the paper with the following remark:
\begin{quote}
``The difference between the bounds increases rapidly and it is necessary to
sharpen them. One can expect an improvement mainly for the upper bound.''
\end{quote}
The conjecture in its present-day form was formulated a bit later, after the
expectation in the above quotation was confirmed by \cite{Starke:1966}.
(Namely, Starke improved the upper bound in~\eqref{first bound} to
$1+\frac{n(n-1)(n-2)}2$, which was the first polynomial upper bound for
$\mathfrak{C}(n)$.) \v{C}ern\'{y} explicitly stated the conjecture
$\mathfrak{C}(n)=(n-1)^2$ in his talks in the second half of the 1960s; in
print the conjecture first appeared in~\cite{Cerny&Piricka&Rosenauerova:1971}.

The best upper bound for the \v{C}ern\'{y} function achieved so far guarantees
that for every \san\ with $n$ states there exists a reset word of length
$\frac{n^3-n}6$. Such a reset word arises as the output of the following greedy
algorithm.

%=================================
\begin{algorithm}
  \setlength{\commentspace}{6cm}
  \begin{algorithmic}[1]
    \FUNC{GreedyCompression$(\mathcal{A})$}
    \STATE\algcomment{0}{Initializing the current word}$w\gets\varepsilon$
    \STATE\algcomment{0}{Initializing the current set}$P\gets Q$
    \WHILE{$|P|> 1$}
    \IF{$|P\cdot u|=|P|$ for all $u\in A^*$}
    \RETURN Failure
    \ELSE\STATE{}take a word $v\in A^*$ of minimum length with $|P\cdot v|<|P|$
    \ENDIF
    \STATE\algcomment{1}{Updating the current word}$w\gets wv$
    \STATE\algcomment{1}{Updating the current set}$P\gets P\cdot v$
    \ENDWHILE
    \RETURN $w$
  \end{algorithmic}
\caption{Greedy algorithm finding a \sw\ for a given DFA
$\mathcal{A}=(Q,A)$}\label{KV:Greedy}
\end{algorithm}
%==================================

If $|Q|=n$, then clearly the main loop of Algorithm~\ref{KV:Greedy} is executed
at most $n-1$ times. In order to evaluate the length of the output word $w$, we
estimate the length of each word $v$ produced by the main loop.

Consider a generic step at which $|P|=k>1$ and let $v=a_1\cdots a_\ell$ with
$a_i\in\Sigma$, $i=1,\dots,\ell$. Then it is easy to see that the sets
$$P_1=P,\ P_2=P_1\cdot a_1,\ \dotsc,\ P_\ell=P_{\ell-1}\cdot a_{\ell-1}$$
are $k$-element subsets of $Q$. Furthermore, since $|P_{\ell}\cdot
a_{\ell}|<|P_\ell|$, there exist two distinct states $q_\ell,q'_\ell\in P_\ell$
such that $q_{\ell}\cdot a_{\ell}=q'_{\ell}\cdot a_{\ell}$. Now define couples
$R_i=\{q_i,q'_i\}\subseteq P_i$, $i=1,\dots,\ell$, such that $q_i \cdot
a_i=q_{i+1}$, $q'_i \cdot a_i=q'_{i+1}$ for $i=1,\dots,\ell-1$.
\begin{figure}[ht]
\begin{center}
\unitlength=.85mm
\begin{picture}(85,37)\nullfont
\drawoval(0,18,15,38,5) \drawoval(25,18,15,38,5) \drawoval(80,18,15,38,5)
\gasset{Nframe=n,Nfill=n} \node(A)(0,10){$q_1$} \node(B)(0,26){$q'_1$}
\node(C)(25,10){$q_2$} \node(D)(25,26){$q'_2$} \node(E)(80,10){$q_\ell$}
\node(F)(80,26){$q'_\ell$} \node[Nframe=y](G)(105,18){} \drawedge(A,C){$a_1$}
\drawedge(B,D){$a_1$} \drawedge(E,G){$a_\ell$} \drawedge(F,G){$a_\ell$}
\put(-13,2){$P_1$} \put(12,2){$P_2$} \put(67,2){$P_\ell$} \node(C1)(50,10){}
\node(D1)(50,26){} \node(E0)(55,10){} \node(F0)(55,26){} \drawedge(C,C1){$a_2$}
\drawedge(D,D1){$a_2$} \drawedge(E0,E){$a_{\ell-1}$}
\drawedge(F0,F){$a_{\ell-1}$} \put(51,10){\dots} \put(51,26){\dots}
\end{picture}
\end{center}
\caption{Combinatorial configuration at a generic step of
Algorithm~\ref{KV:Greedy}}\label{KV:fig:combinatorial configuration}
\end{figure}

\noindent Then the condition that $v$ is a word of minimum length with $|P\cdot
v|<|P|$ implies that $R_i\nsubseteq P_j$ for $1\le j<i\le\ell$. Indeed, if
$R_i\subseteq P_j$ for some $j<i$, then already the word $a_1\cdots
a_ja_i\cdots a_\ell$ of length $j+\ell-i<\ell$ would satisfy $|P\cdot a_1\cdots
a_ja_i\cdots a_\ell|<|P|$ contradicting the choice of $v$. Thus, we arrive at a
problem from combinatorics of finite sets that can be stated as follows. Let
$1<k\le n$. A sequence of $k$-element subsets $P_1,P_2,\dots$ of an $n$-element
set is called 2-\emph{renewed}\index{2-renewed sequence} if each $P_i$ contains
a couple $R_i$ such that $R_i\nsubseteq P_j$ for each $j<i$. What is the
maximum length of a 2-renewed sequence as a function of $n$ and $k$?

The problem was solved by Frankl~\cite{Frankl:1982} who proved the following
result\footnote{Actually Frankl~\cite{Frankl:1982} considered and solved a more
general problem concerning the maximum length of (analogously defined)
$m$-renewed sequences of $k$-element subsets in an $n$-element set for an
arbitrary fixed $m\le k$.}.
\begin{proposition}
\label{KV:prop:frankl} The maximum length of a $2$-renewed sequence of
$k$-element subsets in an $n$-element set is equal to $\binom{n-k+2}2$.
\end{proposition}
Thus, if $\ell_k$ is the length of the word $v$ that Algorithm~\ref{KV:Greedy}
appends to the current word $w$ after the iteration step that the algorithm
enters while the current set $P$ contains $k$ states, then
Proposition~\ref{KV:prop:frankl} guarantees that $\ell_k\le\binom{n-k+2}2$.
Summing up all these inequalities from $k=n$ to $k=2$, one arrives at the
aforementioned bound
\begin{equation}
\label{pin} \mathfrak{C}(n)\le\dfrac{n^3-n}6.
\end{equation}
In the literature the bound~\eqref{pin} is usually attributed to Pin who
explained the above connection between Algorithm~\ref{KV:Greedy} and the
combinatorial problem on the maximum length of 2-renewed sequences and
conjectured the estimation $\binom{n-k+2}2$ for this length in his talk at the
Colloquium on Graph Theory and Combinatorics held in Marseille in 1981. (Frankl
learned this conjecture from Pin---and proved it---during another colloquium on
combinatorics held in Bielefeld in November 1981.) Accordingly, the usual
reference for~\eqref{pin} is the paper \cite{Pin:1983} based on the talk. The
full story is however more complicated. Actually, the bound~\eqref{pin} first
appeared in~\cite{Fischler&Tannenbaum:1970} where it was deduced from a
combinatorial conjecture equivalent to Pin's one. The conjecture however
remained unproved. The bound~\eqref{pin} then reoccurred
in~\cite{Kohavi&Winograd:1971,Kohavi&Winograd:1973} but the argument justifying
it in these papers was insufficient. In 1987 both~\eqref{pin} and
Proposition~\ref{KV:prop:frankl} were independently rediscovered by Klyachko,
Rystsov and Spivak~\cite{Klyachko&Rystsov&Spivak:1987} who were aware
of~\cite{Fischler&Tannenbaum:1970,Kohavi&Winograd:1971,Kohavi&Winograd:1973}
but
neither of~\cite{Pin:1983} nor of~\cite{Frankl:1982}.\marginpar{\textbf{If space\\
permits!!}} We sketch here a proof of Frankl's result
following~\cite{Klyachko&Rystsov&Spivak:1987}.

\begin{proof}[Proof of Proposition~\ref{KV:prop:frankl}] Let $Q=\{1,2,\dots,n\}$.
First, we exhibit a 2-renewed sequence of $k$-element subsets in $Q$ of length
$\binom{n-k+2}2$. For this put $W=\{1,\dots,k-2\}$, list all $\binom{n-k+2}2$
couples of $Q\setminus W$ in some order and let $T_i$ be the union of $W$ with
the $i$-th couple in the list. Clearly, the sequence
$T_1,\dots,T_{\binom{n-k+2}2}$ is 2-renewed.

Now we assign to each $k$-element subset $S=\{s_1,\dots,s_k\}$ of $Q$ the
following polynomial $D(S)$ in variables $x_{s_1},\dots,x_{s_k}$ over the field
of rationals:
$$D(S)=\begin{vmatrix}
1 & s_1 & s_1^2 & \cdots & s_1^{k-3} & x_{s_1} & x_{s_1}^2\\
1 & s_2 & s_2^2 & \cdots & s_2^{k-3} & x_{s_2} & x_{s_2}^2\\
\vdots & \vdots & \vdots & \ddots & \vdots & \vdots & \vdots\\
1 & s_k & s_k^2 & \cdots & s_k^{k-3} & x_{s_k} & x_{s_k}^2
\end{vmatrix}_{k\times k}.$$
Observe that for any 2-renewed sequence $P_1,\dots,P_\ell$ of $k$-element
subsets in $Q$, the polynomials $D(P_1),\dots,D(P_\ell)$ are linearly
independent. Indeed, if they were linearly dependent, then by a basic lemma of
linear algebra, some polynomial $D(P_j)$ should be expressible as a linear
combination of the preceding polynomials $D(P_1),\dots,D(P_{j-1})$. By the
definition of a 2-renewed sequence, $P_j$ contains a couple $\{s,s'\}$ such
that $\{s,s'\}\nsubseteq P_i$ for all $i<j$. If we substitute $x_s=s$,
$x_{s'}=s'$ and $x_t=0$ for $t\ne s,s'$ in each polynomial
$D(P_1),\dots,D(P_j)$, then the polynomials $D(P_1),\dots,D(P_{j-1})$ vanish
(since the two last columns in each of the resulting determinants become
proportional) and so does any linear combination of the polynomials. The value
of $D(P_j)$ however is the determinant being the product of a Vandermonde
$(k-2)\times(k-2)$-determinant with $\begin{vmatrix} s & s^2\\ s' &
(s')^2\end{vmatrix}$, and thus, it is not 0. Hence $D(P_j)$ cannot be equal to
a linear combination of $D(P_1),\dots,D(P_{j-1})$.

We see that the length of any 2-renewed sequence cannot exceed the dimension of
the linear space over the field of rationals spanned by all polynomials of the
form $D(S)$. In order to prove that the dimension is at most $\binom{n-k+2}2$,
it suffices to show that the space is spanned by the polynomials
$D(T_1),\dots,D(T_{\binom{n-k+2}2})$ where $T_1,\dots,T_{\binom{n-k+2}2}$ is
the 2-renewed sequence constructed in the first paragraph of the proof.
\end{proof}

\section{The road coloring problem}
\label{KV:sec:rcp}

\section{Generalizations}

\bibliographystyle{abbrv}
\addcontentsline{toc}{section}{References}
\begin{footnotesize}
  \bibliography{abbrevs,SA}
\end{footnotesize}


\addcontentsline{toc}{section}{Index}
\markright{\indexname}\markboth{\indexname}{\indexname}
\printindex

\end{document}
